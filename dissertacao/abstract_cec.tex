The concept of example credibility evaluates how much a classifier can trust an example when building a classification model.
It is given by a credibility function, which is application dependent and estimated according to a series of factors that influence the credibility of the examples.
In this work, we deal with automatic classification and study the credibility of an example according to two factors: its attributes and relationships. 
We propose a genetic programming algorithm to estimate the credibility of examples, and then add this estimation to a credibility-aware classifier. 
For that, we model the relationship credibility as a complex network and the attribute credibility using a set of feature selection metrics. These metrics are used as terminals for the GP. The GP was tested in a large number of databases and results showed that the credibility-aware classifier obtained significant improvements of micro and macroF$_1$ compared to the traditional classifiers.




O conceito dos exemplos de credibilidade avalia o quanto um classificador pode confiar nos exemplos quando constrói um modelo de classificação.
Isto é dado por uma função de credibilidade, que é dependente da aplicação e é estimada de acordo com uma série de fatores que influenciam a credibilidade dos exemplos.
Nesse trabalho, lidamos com classificação automática e estudamos a credibilidade de um exemplo de acordo com dois fatores: o seus atributos e relações.
Nos propomos um algoritmo baseado em Programação Genética (PG) para estimar a credibilidade dos exemplos, e então adicionamos esta estimativa aos classificadores.
Para tanto, modelamos a credibilidade baseada em relacionamentos como uma rede complexa e a credibilidade dos atributos usando um conjunto de métricas de seleção de atributos.
O PG foi testado em um grande número de bases de dados e os resultados mostram que um classificador que leva em conta a credibilidade obtém resultados de micro e macro$F_1$ significativamente melhores quando comparados a classificadores tradicionais.
