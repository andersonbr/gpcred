
O conceito dos exemplos de credibilidade avalia o quanto um classificador pode confiar nos exemplos quando constrói um modelo de classificação.
Isto é dado por uma função de credibilidade, que é dependente da aplicação e é estimada de acordo com uma série de fatores que influenciam a credibilidade dos exemplos.
Nesse trabalho, lidamos com classificação automática e estudamos a credibilidade de um exemplo de acordo com dois fatores: o seus atributos e relações.
Nos propomos um algoritmo baseado em Programação Genética (PG) para estimar a credibilidade dos exemplos, e então adicionamos esta estimativa aos classificadores.
Para tanto, modelamos a credibilidade baseada em relacionamentos como uma rede complexa e a credibilidade dos atributos usando um conjunto de métricas de seleção de atributos.
O PG foi testado em um grande número de bases de dados e os resultados mostram que um classificador que leva em conta a credibilidade obtém resultados de micro e macro$F_1$ significativamente melhores quando comparados a classificadores tradicionais.

\keywords{Classificação automática, Programação Genética, Credibilidade}
