
\section{Programação Genética}

\subsection{Operadores}

\subsection{Fitness}


\section{Credibilidade baseada no Conteúdo}
\label{sec::pg_cred_baseada_conteudo}

\subsection{Métricas Modeladas.}
\label{subsec::pg_metricas_conteudo}


%%%%%%%%%%%%%%%%%%%-------------------------------------------------------%%%%%%%%%%%%%%%%%%%%%%%%%%%%%%%%%%%%%%%%
\subsubsection{Medida de Ambiguidade - Ambiguity Measure (AM)}
\label{subsubsection::am}

A medida de ambiguidade (AM de \textit{Ambiguity Measure}) foi definida por~\cite{Mengle08} e utilizada como um método de seleção de atributos. Ele atribui valores maiores para os atributos considerados menos ambíguos, onde um atributo é não ambíguo quando sua presença indica com um alto grau de confiança que o exemplo pertence a uma classe específica. Dado que $N_{x_{i}c{j}}$ é o número de vezes que temos o valor $x_i$ para o atributos $A_i$ na classe $c_j$, podemos calcular $AM(x_i, c_j)$ como:
\begin{equation}\label{eqn::am}
   AM(x_i, c_j) = \frac{ N_{x_{i}c_{j}}}{\sum\limits_{c_k \in \mathbb{C}} N_{x_{i}c_{k}}}.
\end{equation}

%%%%%%%%%%%%%%%%%%%-------------------------------------------------------%%%%%%%%%%%%%%%%%%%%%%%%%%%%%%%%%%%%%%%%
\subsubsection{Probabilidade Condicional}
\label{subsubsection::pc}

A probabilidade condicional $P(x_i|c_j)$ provém do algoritmo \textit{Naïve Bayes} como foi discutido na Seção~\ref{subsec::cred_nb}.
Temos dois modos de calcular $P(x_i|c_j)$, um para quando temos $A_i$ categórico e outro para quando estamos realizando classificação textual.
Basicamente, a ideia de ambos é a mesma, para uma dada classe $c_j$, calculamos a probabilidade do atributo $A_i$ ter o valor $x_i$ entre todos possíveis. 
Para classificação categórica, basta apenas contar quantas vezes $A_i$ vale $x_i$ para uma dada classe $c_j$:
    \begin{equation}\label{eqn::nbcattexto}
        P(x_i|c_j) = \frac{ N_{x_{i}c_{j}} }{ N_{c_{j}} } 
    \end{equation}
Onde $N_{x_{i}c_{j}}$ é o número de vezes que temos o termo $x_i$ nos exemplos de treino da classe $c_j$ e $N_{c_{j}}$ é o número de exemplos de treino da classe $c_j$.
        
Para classificação textual, contamos quantas vezes um termo aparece em uma classe em comparação a todos os termos possíveis:
    \begin{equation}\label{eqn::nbcattexto}
        P(x_i|c_j) = \frac{ N_{x_{i}c_{j}} }{ \sum\limits^{D}_{k = 1} {  N_{x_{k}c_{j}}} } 
    \end{equation}
Onde $N_{x_{i}c_{j}}$ é o número de vezes que temos o termo $x_i$ nos documentos de treino da classe $c_j$ e $D$ é o número o vocabulário conhecido no treino.

%%%%%%%%%%%%%%%%%%%-------------------------------------------------------%%%%%%%%%%%%%%%%%%%%%%%%%%%%%%%%%%%%%%%%
\subsubsection{Inverso da Probabilidade Condicional}
\label{subsubsection::pc'}
Com o inverso da probabilidade condicional, calculamos a probabilidade de um atributo $A_i$ não valer $x_i$ para uma classe $c_j$. Podemos realizar esse calculo apenas com a seguinte fórmula:
\begin{equation}\label{eqn::plinhattalquec}
   P(\overline{x_i}|c_j) = 1.0 - P(x_i|c_j)
\end{equation}

%%%%%%%%%%%%%%%%%%%-------------------------------------------------------%%%%%%%%%%%%%%%%%%%%%%%%%%%%%%%%%%%%%%%%
\subsubsection{Índice de Gini melhorado}
\label{subsubsection::gini}

O índice de Gini é uma métrica baseada na curva de Lorenz que mostra a função de distribuição acumulada de uma variável. Esse índice é amplamente utilizado nas Ciências Econômicas como uma métrica para avaliação da distribuição de renda pela população de um certo país ou região. Infelizmente por essa métrica, o Brasil é um dos países mais desiguais do mundo (ver \cite{cia-gini}). 
Baseado na ideia de desigualdade, podemos pensar na distribuição de um atributo nas M classes possíveis. Um atributo que seja desigualmente distribuído é certamente um atributo com um maior poder de discriminação, e portanto, um atributo mais importante. O trabalho de~\cite{Shang07} consistiu na criação de um selecionador de atributos para classificadores textuais baseando-se no Índice de Gini, chamado Índice de Gini melhorado. Ao contrario da maioria das métricas expostas nessa seção, o Índice de Gini melhorado tem apenas um parâmetro, o valor do $i$-ésimo atributo, não levando em consideração nenhuma classe específica. Ele é considerado melhorado por algumas pequenas diferenças com o método tradicional de Gini, entre elas o fato de um maior valor se referir a um melhor atributo e não ao contrário como é feito no método original. A fórmula sugerida por \cite{Shang07} para ser utilizada em classificação de atributos textuais é dada por:
\begin{equation}\label{eqn::gini}
   GINI(x_i) = \sum\limits_{c_k \in \mathbb{C}} P(x_i|c_k)^2 \cdot P(c_k|x_i)^2
\end{equation}
Shang et al. destaca o fato da não utilização do fator $P(x_i)$, fazendo com que o Índice de GINI melhorado sofra menos a influência de atributos frequentes conseguindo capturar a capacidade de um atributo ser importante para distinguir uma classe, independente de qual classe é. Destacamos que $P(c|x_i)$ é justamente a probabilidade que o algoritmo Bayesiano pretende calcular, logo aproximamos esse fator como:
\begin{equation}\label{eqn::gini}
   P(c_j|x_i) = \frac{  P(c_j \wedge x_i) } { P(x_i) } = \frac{ \frac{ N_{x_ic_j}}{  \sum\limits_{c \in \mathbb{C}} \sum\limits_{k=1}^{D} {Nx_kc}  } } { \frac{\sum\limits_{c \in \mathbb{C}} N_{x_ic}}{ \sum\limits_{c \in \mathbb{C}} \sum\limits_{k=1}^{D} {Nx_kc}}} = \frac{ N_{x_{i}c_{j}}}{\sum\limits{c_k \in \mathbb{C}} N_{x_{i}c_{k}}}.
\end{equation}
Que é o mesmo valor definido por~\cite{Mengle08} para a métrica Medida da Ambiguidade mostrada na Seção~\ref{subsubsection::am}.

%%%%%%%%%%%%%%%%%%%-------------------------------------------------------%%%%%%%%%%%%%%%%%%%%%%%%%%%%%%%%%%%%%%%%
\subsubsection{Ganho de Informação}
\label{subsubsection::ig}

O Ganho de Informação (\textit{Information Gain}, IG), \cite{Yang97}, mede a diminuição da entropia quando um atributo é usado ou não. A entropia é uma medida utilizada no campo da Ciência da Informação que tenta quantificar a desordem, a imprevisibilidade. Quanto maior a entropia maior a entropia, mais difícil é prever um resultado, portanto a métrica IG seleciona os atributos que diminuam o valor da entropia, facilitando descobrir a qual classe um exemplo pertence. O Ganho da Informação pode ser calculado da seguinte forma:
\begin{equation}\label{eqn::ig}
   IG(x_i, c_j) = \sum_{c \in \{c_j, \overline{c_j}\}}\sum_{x \in \{x_i, \overline{x_i}\}}P(x|c)\log_2\frac{P(x|c)}{P(x)P(c)}.
\end{equation}

%%%%%%%%%%%%%%%%%%%-------------------------------------------------------%%%%%%%%%%%%%%%%%%%%%%%%%%%%%%%%%%%%%%%%
\subsubsection{Cross Entropy}
\label{subsubsection::}

\textit{Cross Entropy} (CE), assim como o Índice de Gini Melhorado, Seção~\cite{subsubsection::gini}, apresenta apenas um parâmetro, um atributo. Novamente aproximamos o calculamos $P(c|x_i)$, pois esse já é o resultado do algoritmo \textit{Naïve Bayes} e, portanto, não o teríamos enquanto estamos calculando a credibilidade de um atributo em relação a uma classe. Assim como enunciado por~\cite{Koller97} e adaptado para seleção de atributos em~\cite{Mladenic98}, essa métrica assume a seguinte fórmula:
\begin{equation}\label{eqn::ce}
   CE(x_i) =  P(x_i) \cdot \sum_{c \in \mathbb{C}} P(c|x_i) \cdot \log_2 \frac{ P(c|x_i) } { P(c) }
\end{equation}

%%%%%%%%%%%%%%%%%%%-------------------------------------------------------%%%%%%%%%%%%%%%%%%%%%%%%%%%%%%%%%%%%%%%%
\subsubsection{CHI-Quadrado}
\label{subsubsection::chi}

O teste \textit{CHI-quadrado - $\chi^2$} é utilizado no campo da Estatística para testar a independência entre dois eventos. Quando usado para seleção de atributos, tipicamente temos que os dois eventos são a ocorrência de uma classe e a ocorrência de um atributo $A_i$ com valor $x_i$, ver~\cite{Zheng03}. Logo,
\begin{equation}\label{eqn::chi}
   CHI(x_i, c_j) = N \cdot \frac{ [ P(x_i|c_j) \cdot P(\overline{x_i}|\overline{c_j}) - P(x_i|\overline{c_j}) \cdot P(\overline{x_i}|c_j) ]^2 } {P(x_i) \cdot P(\overline{x_i}) \cdot P(c_j) \ \cdot P(\overline{c_j}) }
\end{equation}
como $P(x_i|c_j)$ a probabilidade de termos o atributo $A_i$ com valor $x_i$ dada a classe $c_j$ para classificação categórica e a probabilidade de um documento ter o termos $x_i$ dado que ele pertence a classe $c_j$. Valores próximos de zero indicam a falta de relação entre $x_i$ e $c_j$.

%%%%%%%%%%%%%%%%%%%-------------------------------------------------------%%%%%%%%%%%%%%%%%%%%%%%%%%%%%%%%%%%%%%%%
\subsubsection{Coeficiente de Correlação}
\label{subsubsection::cc}

O Coeficiente de Correlação (CC) é uma métrica de seleção de atributos, variante da métrica \textit{CHI-Quadrada - $\chi^2$}, ver Seção~\ref{subsubsection::chi}. Definida por~\cite{Ng97}, temos que $(CC)^2 = \chi^2$, logo:
\begin{equation}\label{eqn::ce}
   CC(x_i, c_j) = \sqrt{N} \cdot \frac{ P(x_i|c_j) \cdot P(\overline{x_i}|\overline{c_j}) - P(x_i|\overline{c_j}) \cdot P(\overline{x_i}|c_j) } {\sqrt{ P(x_i) \cdot P(\overline{x_i}) \cdot P(c_j) \ \cdot P(\overline{c_j}) } }
\end{equation}
como a mesma definição de $P(x_i|c_j)$ usada para a métrica $\chi^2$. Os valores positivos para CC correspondem a pertinência do valor de um atributo a uma classe, enquanto valores negativos indicam a não pertinência. Quanto mais positivo (negativo) são os valores de CC, mais fortemente o atributo é relacionado (não relacionado) a uma classe. Para fins de seleção de atributo, valores mais elevados de CC são os mais interessantes, pois mostram uma correlação positiva entre um atributo e uma classe. Em contraste com CC, $\chi^2$ também considera importantes correlações negativas entre atributos e classes, o que acaba resultando que atributos que fortemente indicam a falta de pertinência a uma classe são tão importantes quanto os que indicam.

%%%%%%%%%%%%%%%%%%%-------------------------------------------------------%%%%%%%%%%%%%%%%%%%%%%%%%%%%%%%%%%%%%%%%
\subsubsection{Coeficiente GSS}
\label{subsubsection::gss}
O coeficiente Galavotti–Sebastiani–Simi (GSS) introduzido por~\cite{Galavotti00} é bastante similar a $\chi^2$, Seção~\ref{subsubsection::chi}, e pode ser definido como:
\begin{equation}\label{eqn::gss}
   GSS(x_i, c_j) = P(x_i|c_j) \cdot P(\overline{x_i}|\overline{c_j}) - P(x_i|\overline{c_j}) \cdot P(\overline{x_i}|c_j) 
\end{equation}
sendo $P(x_i|c_j)$ definido na Seção~\ref{subsubsection::pc}. Da mesma forma como o Coeficiente de Correlação, Seção~\ref{subsubsection::cc}, valores positivos correspondem à pertinência de um atributo a uma categoria e, negativos, à não pertinência. 


%%%%%%%%%%%%%%%%%%%-------------------------------------------------------%%%%%%%%%%%%%%%%%%%%%%%%%%%%%%%%%%%%%%%%
\subsubsection{Razão de Chances - \textit{Odds Ratio} (OR)}
\label{subsubsection::or}

Proposta originalmente por~\cite{Rijsbergen79}, a métrica Razão de Chances, do inglês \textit{Odds Ration} (OR), também é uma métrica de seleção de atributos. A ideia básica é que a distribuição de atributos em exemplos relevantes é diferente da distribuição de atributos em exemplos não relevantes. Isso quer dizer que poderíamos definir dois eventos A e B e calculamos a probabilidade de ocorrência de A dividida pela probabilidade de não ocorrência de A e a comparamos com a probabilidade ocorrência de B dividida pela probabilidade de não ocorrência de B:
\begin{equation}\label{eqn::or}
   OR(A, B) = \frac{\frac{A}{1-A}} {\frac{B}{1-B}} = \frac{ A \cdot ( 1 - B )} { B \cdot ( 1 - A ) } 
\end{equation}
Uma razão de chances de 1.0 indica que ocorrer A ou B é igualmente provável, uma razão maior do que 1.0 indica que ocorrer A é mais provável, enquanto que uma razão de chances menor do que 1 indica que o evento B é tem uma probabilidade maior de ocorrer.

A razão de chances tem sido utilizada para selecionamento de atributos por \cite{Mladenic98} fazendo com que A seja $P(x_i|c_j)$ e B seja $P(x_i,\overline{c_j})$. Logo,
\begin{equation}\label{eqn::or}
   OR(x_i, c_j) = \frac{ P(x_i|c_j) \cdot [ 1.0 - P(x_i|\overline{c_j}) ] }{ [ 1.0 - P(x_i|c_j) ] \cdot P(x_i|\overline{c_j})},
\end{equation}
em que $P(x_i, c_j)$ é a probabilidade de termos o atributo $A_i$ com valor $x_i$ em exemplos que pertencem a classe $c_j$ e $P(x_i|\overline{c_j})$ é a probabilidade de termos $x_i$ em outras classes que não sejam $c_j$.

%%%%%%%%%%%%%%%%%%%-------------------------------------------------------%%%%%%%%%%%%%%%%%%%%%%%%%%%%%%%%%%%%%%%%
\subsubsection{Frequência dos Documentos por Termo}
\label{subsubsection::df}

\begin{equation}\label{eqn::df}
   DF(x_i) = 1.0 + \log{ ( DF_{x_i} ) }
\end{equation}


%%%%%%%%%%%%%%%%%%%-------------------------------------------------------%%%%%%%%%%%%%%%%%%%%%%%%%%%%%%%%%%%%%%%%

\subsubsection{Frequência de Docummentos por Termo-Classe}
\label{subsubsection::sumdf}

\begin{equation}\label{eqn::sumtf}
   DFporClasse(x_i) = 1.0 + \log{ ( DF_{x_ic_k} ) }
\end{equation}


%%%%%%%%%%%%%%%%%%%-------------------------------------------------------%%%%%%%%%%%%%%%%%%%%%%%%%%%%%%%%%%%%%%%%
\subsubsection{Inverso da Frequência de Documentos (IDF)}
\label{subsubsection::idf}

O Inverso da Frequência de Documentos, IDF do inglês \textit{Inverse Document Frequency}, é uma conhecida métrica que avalia a popularidade de um determinado termo em um conjunto de documentos. Temos que quanto mais popular é um termo ao longo dos documentos de uma coleção, pior é sua capacidade de discriminação, logo:
\begin{equation}\label{eqn::tficf}
   IDF(x_i) = log( \frac{|N|} {DF(x_i)} )
\end{equation}
Onde temos $N$ documentos na base de treinamento da coleção e $DF_{x_i}$ é o número de documentos que apresentam o termo $x_i$.



%%%%%%%%%%%%%%%%%%%-------------------------------------------------------%%%%%%%%%%%%%%%%%%%%%%%%%%%%%%%%%%%%%%%%
\subsubsection{Frequência do Termo}
\label{subsubsection::sumtf}
\begin{equation}\label{eqn::sumtf}
   TF(x_i) = 1.0 + \log{ ( N_{x_i} ) }
\end{equation}


%%%%%%%%%%%%%%%%%%%-------------------------------------------------------%%%%%%%%%%%%%%%%%%%%%%%%%%%%%%%%%%%%%%%%
\subsubsection{Frequência do Termo por Classe}
\label{subsubsection::tf}
TF per class...
\begin{equation}\label{eqn::tf}
   TFporClasse(x_i, c_j) = 1.0 + \log{ ( N_{x_ic_j} ) }
\end{equation}

%%%%%%%%%%%%%%%%%%%-------------------------------------------------------%%%%%%%%%%%%%%%%%%%%%%%%%%%%%%%%%%%%%%%%

\subsubsection{TFIDF}
\label{subsubsection::tfidf}

Uma das métricas de pesos para atributos em classificação de textos mais populares na literatura é o TFIDF (do inglês, \textit{Term Frequency Inversed Document Frequency}), ver \cite{Salton88}. O TFIDF combina a frequência de um termo (\textit{Term Frequency}) que assume que múltiplas aparições de um termo em um documento é mais importante que aparições únicas com o inverso da frequência de um documento (\textit{Inverded Document Frequency}) que diz que termos raros são de maior poder discriminativo que termos muito frequentes. Em síntese, a fórmula de TFIDF é nada mais que a multiplicação das métricas TF e IDF, como já esperávamos:
\begin{equation}\label{eqn::tficf}
   TFIDF(x_i, c_j) =  N_{x_ic_j} \cdot log( \frac{DF_{c_j}}{ DF_{x_ic_j} } )
\end{equation}
Onde $N_{x_ic_j}$ é o número de vezes que temos o termo $x_i$ na classe $c_j$, $DF_{c_j}$ é o número de documentos da classe $c_j$ e, finalmente, $DF_{x_ic_j}$ é o número de documentos que contém o termo $x_i$ e são da classe $c_j$.

%%%%%%%%%%%%%%%%%%%-------------------------------------------------------%%%%%%%%%%%%%%%%%%%%%%%%%%%%%%%%%%%%%%%%
\subsubsection{TFICF}
\label{subsubsection::tficf}

O TFICF (do inglês, \textit{Term Frequency Inversed Class Frequency}) é uma variação do esquema TFIDF (ver Seção~\ref{subsubsection::tfidf}). Novamente, TF se refere a quanto importante é um termo em uma classe, pois trata-se de sua frequência; por sua vez, ICF é interpretado como a frequência relativa de um termo em uma classe. Como é possível observar, não existe uma forma de diferenciarmos entre um termo que aparece frequentemente em um pequeno subconjunto de documentos e um termo que está presente em um grande número de documentos entre as classes. Como mostrado por~\cite{ChihHow04}, a fórmula para TFICF é:
\begin{equation}\label{eqn::tficf}
   TFICF(x_i, c_j) = N_{x_ic_j} \cdot log( \frac{M}{CF_{x_i}} )
\end{equation}
Onde temos que $N_{x_ic_j}$ é o número de vezes que temos o termo $x_i$ na classe $c_j$, $M$, como previamente definido, é o número de classes da coleção e $CF_{x_i}$ é o número de classes que contém o termo $x_i$. 

%%%%%%%%%%%%%%%%%%%-------------------------------------------------------%%%%%%%%%%%%%%%%%%%%%%%%%%%%%%%%%%%%%%%%
\subsubsection{Category Term Description}
\label{subsubsection::ctd}

Definido por~\cite{ChihHow04}, \textit{Category Term Description} é uma métrica de seleção de atributos para classificação textual baseada em TFIDF (Seção~\ref{subsubsection::tfidf}) e TFICF (Seção~\ref{subsubsection::tficf}). How et al. propõe uma melhoria ao TDICF acreditando que quanto menos um termo ocorre entre os documentos, maior o poder discriminativo daquele termo, logo:
\begin{equation}\label{eqn::cdt}
   CDT(x_i, c_j) = TF(x_i, c_j) \cdot ICF(x_i, c_j) \cdot IDF(x_i,c_j) =  N_{x_ic_j} \cdot log( \frac{M}{CF_{x_i}} ) \cdot  log( \frac{DF_{c_j}}{ DF_{x_ic_j} } )
\end{equation}
Onde temos que $N_{x_ic_j}$ é o número de vezes que temos o termo $x_i$ na classe $c_j$, $M$ é o número de classes da coleção, $CF_{x_i}$ é o número de classes que contém o termo $x_i$, $DF_{c_j}$ é o número de documentos da classe $c_j$ e $DF_{x_ic_j}$ é o número de documentos que contém o termo $x_i$ e são da classe $c_j$.

%%%%%%%%%%%%%%%%%%%-------------------------------------------------------%%%%%%%%%%%%%%%%%%%%%%%%%%%%%%%%%%%%%%%%
\subsubsection{Dominância}
\label{subsubsection::dom}

Dominância é uma métrica originalmente proposta em~\cite{Zaiane02} e utilizada mais recentemente no trabalho de~\cite{Rocha08} para criação do que foi chamado de janela temporal de um documento. Em suma, utilizado exclusivamente em classificação textual, o método normaliza a frequência de um documento em um classe por todas as classes existentes, logo:

\begin{equation}\label{eqn::dom}
   Dominancia(x_i, c_j) = \frac{ DF_{x_ic_j} } { \sum\limits_{c_k \in \mathbb{C}} DF_{ x_ic_k } } 
\end{equation}




\subsubsection{MaxDom}
\label{subsubsection::maxdom}

\subsubsection{MaxIG}
\label{subsubsection::maxig}

\subsubsection{MaxAM}
\label{subsubsection::maxam}

\subsubsection{MaxCHI}
\label{subsubsection::maxchi}

\subsubsection{MaxGSS}
\label{subsubsection::maxgss}

\subsubsection{MaxCC}
\label{subsubsection::maxcc}

\subsubsection{MaxOR}
\label{subsubsection::maxor}

\subsubsection{MaxCTD}
\label{subsubsection::maxctd}

\subsubsection{MaxTFIDF}
\label{subsubsection::maxtfidf}

\subsubsection{MaxTFICF} 
\label{subsubsection::maxtficf}

\cite{ChihHow04}

\section{Credibilidade baseada em Grafos}
\label{sec::pg_cred_baseada_grafos}

\subsection{Métricas modeladas.}
\label{subsec::pg_metricas_grafos}

asdf


\subsubsection{NeighborhoodSize3(id,className,graphId)}
break;
\subsubsection{getHubScore(id, className, graphId)}
break;
\subsubsection{getAuthority(id, className, graphId)}
break;
\subsubsection{getEigenVectorCentrality(id, className, graphId)}
break;
\subsubsection{getCloseness(id, className, graphId)}
break;
\subsubsection{getStrength(id, className, graphId)}
break;
\subsubsection{getConstraint(id, className, graphId)}
break;
\subsubsection{getPageRank(id, className, graphId)}
break;
\subsubsection{getBetweenness(id, className, graphId)}
break;
\subsubsection{getBibCoupling(id, className, graphId)}
break;
\subsubsection{getCoCitation(id, className, graphId)}
break;
\subsubsection{getJaccardSimilarity(id, className, graphId)}
break;
\subsubsection{getDiceSimilarity(id, className, graphId)}
break;
\subsubsection{getInverseLogSimilarity(id, className, graphId)}
break;
\subsubsection{getAvgNearstNeighborDegree(id, className, graphId)}
acho que nao to uando

