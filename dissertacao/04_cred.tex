
\section{Programação Genética}

\subsection{Operadores}

\subsection{Fitness}


\section{Credibilidade baseada no Conteúdo}
\label{sec::pg_cred_baseada_conteudo}

\subsection{Metricas Modeladas.}
\label{subsec::pg_metricas_conteudo}


%%%%%%%%%%%%%%%%%%%-------------------------------------------------------%%%%%%%%%%%%%%%%%%%%%%%%%%%%%%%%%%%%%%%%
\subsubsection{Medida de Ambiguidade - Ambiguity Measure (AM)}
\label{subsubsection::am}

A medida de ambiguidade (AM de \textit{Ambiguity Measure}) foi definida por~\cite{Mengle08} e utilizada como um método de seleção de atributos. Ele atribui valores maiores para os atributos considerados menos ambíguos, onde um atributo é não ambíguo quando sua preseça indica com um alto grau de confiança que o exemplo pertence a uma classe específica. Dado que $N_{x_{i}c{j}}$ é o número de vezes que temos o valor $x_i$ para o atributos $A_i$ na classe $c_j$, podemos calcular $AM(x_i, c_j)$ como:

\begin{equation}\label{eqn::am}
   AM(x_i, c_j) = \frac{ N_{x_{i}c_{j}}}{\sum_{c_k \in \mathbb{C}} N_{x_{i}c_{k}}}.
\end{equation}

%%%%%%%%%%%%%%%%%%%-------------------------------------------------------%%%%%%%%%%%%%%%%%%%%%%%%%%%%%%%%%%%%%%%%
\subsubsection{Probilidade Condicional}
\label{subsubsection::pc}

A probabilidade condicional $P(x_i|c_j)$ provém do algoritmo \textit{Naïve Bayes} como foi discutido na Seção~\ref{subsec::cred_nb}.
Basicamente, para uma certa classe $c_j$, calculamos a probabilidade do atributo $A_i$ ter o valor $x_i$

\begin{equation}\label{eqn::pctalquec}
    P(x_i|c_j) = \frac{ N_{x_{i}c_{j}} }{ \sum\limits^{d}_{k = 1} { } N_{x_{k}c_{j}} } 
\end{equation}

%%%%%%%%%%%%%%%%%%%-------------------------------------------------------%%%%%%%%%%%%%%%%%%%%%%%%%%%%%%%%%%%%%%%%
\subsubsection{Probabilidade Condicional barrada!?}
\label{subsubsection::pc'}

\begin{equation}\label{eqn::plinhattalquec}
   P'(x_k|C) = 1.0 - P(x_k|c)
\end{equation}

%%%%%%%%%%%%%%%%%%%-------------------------------------------------------%%%%%%%%%%%%%%%%%%%%%%%%%%%%%%%%%%%%%%%%
\subsubsection{Índice de GINI}
\label{subsubsection::gini}

\cite{Shang07}
\begin{equation}\label{eqn::gini}
   GINI(x_k) = \sum_{c \in \mathbb{C}} P(x_k|c)^2 \cdot P(c|x_k)^2
\end{equation}

%%%%%%%%%%%%%%%%%%%-------------------------------------------------------%%%%%%%%%%%%%%%%%%%%%%%%%%%%%%%%%%%%%%%%
\subsubsection{Dominância}
\label{subsubsection::dom}

%%%%%%%%%%%%%%%%%%%-------------------------------------------------------%%%%%%%%%%%%%%%%%%%%%%%%%%%%%%%%%%%%%%%%
\subsubsection{Ganho de Informação}
\label{subsubsection::ig}

\cite{couto06}
\begin{equation}\label{eqn::ig}
   IG(t_k, c_i) = \sum_{c \in \{c_i, \overline{c_i}\}}\sum_{t \in \{t_k, \overline{t_k}\}}P(t|c)\log_2\frac{P(t|c)}{P(t)P(c)}.
\end{equation}

%%%%%%%%%%%%%%%%%%%-------------------------------------------------------%%%%%%%%%%%%%%%%%%%%%%%%%%%%%%%%%%%%%%%%
\subsubsection{CHI}
\label{subsubsection::chi}

\cite{Zheng03}
\begin{equation}\label{eqn::chi}
   CHI(t_k, c_i) = N \cdot \frac{ [ P(x_k|c_i) \cdot P(\overline{x_k}|\overline{c_i}) - P(x_k|\overline{c_i}) \cdot P(\overline{x_k}|c_i) ]^2 } {P(x_k) \cdot P(\overline{x_k}) \cdot P(c_i) \ \cdot P(\overline{c_i}) }
\end{equation}

%%%%%%%%%%%%%%%%%%%-------------------------------------------------------%%%%%%%%%%%%%%%%%%%%%%%%%%%%%%%%%%%%%%%%
\subsubsection{GSS}
\label{subsubsection::gss}

\begin{equation}\label{eqn::gss}
   GSS(t_k, c_i) = P(x_k|c_i) \cdot P(\overline{x_k}|\overline{c_i}) - P(x_k|\overline{c_i}) \cdot P(\overline{x_k}|c_i) 
\end{equation}

%%%%%%%%%%%%%%%%%%%-------------------------------------------------------%%%%%%%%%%%%%%%%%%%%%%%%%%%%%%%%%%%%%%%%
\subsubsection{Coeficiente de Correlação}
\label{subsubsection::cc}
\cite{Zheng03}
\begin{equation}\label{eqn::ce}
   CE(t_k, c_i) = \sqrt{N} \cdot \frac{ P(x_k|c_i) \cdot P(\overline{x_k}|\overline{c_i}) - P(x_k|\overline{c_i}) \cdot P(\overline{x_k}|c_i) } {\sqrt{ P(x_k) \cdot P(\overline{x_k}) \cdot P(c_i) \ \cdot P(\overline{c_i}) } }
\end{equation}

%%%%%%%%%%%%%%%%%%%-------------------------------------------------------%%%%%%%%%%%%%%%%%%%%%%%%%%%%%%%%%%%%%%%%
\subsubsection{Cross Entropy}
\label{subsubsection::}
\cite{Shang07}
\begin{equation}\label{eqn::ce}
   CE(x_k) =  P(x_k) \cdot \sum_{c \in \mathbb{C}} P(c|x_k) \cdot log_2 \frac{ P(c|x_k) } { P(c) }
\end{equation}

%%%%%%%%%%%%%%%%%%%-------------------------------------------------------%%%%%%%%%%%%%%%%%%%%%%%%%%%%%%%%%%%%%%%%
\subsubsection{Odds Ratio}
\label{subsubsection::or}

\cite{Zheng03}
\begin{equation}\label{eqn::or}
   OR(x_k, c_i) = \frac{ P(x_k|c_i) \cdot [ 1.0 - P(x_k|\overline{c_i}) ] }{ [ 1.0 - P(x_k|c_i) ] \cdot P(x_k|\overline{c_i})}
\end{equation}

%%%%%%%%%%%%%%%%%%%-------------------------------------------------------%%%%%%%%%%%%%%%%%%%%%%%%%%%%%%%%%%%%%%%%
\subsubsection{Category Term Description}
\label{subsubsection::ctd}
\cite{ChihHow04}
\begin{equation}\label{eqn::cdt}
   CDT(x_k, c_i) = TF() \cdot IDF() \cdot ICF()
\end{equation}


%%%%%%%%%%%%%%%%%%%-------------------------------------------------------%%%%%%%%%%%%%%%%%%%%%%%%%%%%%%%%%%%%%%%%
\subsubsection{IDF}
\label{subsubsection::idf}

\cite{ChihHow04}
\begin{equation}\label{eqn::tficf}
   IDF(x_k, c_i) = log( \frac{|D|} {df(x_k)} )
\end{equation}


%%%%%%%%%%%%%%%%%%%-------------------------------------------------------%%%%%%%%%%%%%%%%%%%%%%%%%%%%%%%%%%%%%%%%
\subsubsection{TFICF}
\label{subsubsection::tfidf}
\cite{ChihHow04}
\begin{equation}\label{eqn::tficf}
   TFICF(x_k, c_i) = TF(x_k, c_i) \cdot log( \frac{|C|} {cf(x_k)} )
\end{equation}


%%%%%%%%%%%%%%%%%%%-------------------------------------------------------%%%%%%%%%%%%%%%%%%%%%%%%%%%%%%%%%%%%%%%%
\subsubsection{TFIDF}
\label{subsubsection::tfidf}

\subsubsection{TF}
\label{subsubsection::tf}

\subsubsection{sumTF}
\label{subsubsection::sumtf}

\subsubsection{Frequência dos Documentos}
\label{subsubsection::df}


\subsubsection{sumDF}
\label{subsubsection::sumdf}

\subsubsection{MaxDom}
\label{subsubsection::maxdom}

\subsubsection{MaxIG}
\label{subsubsection::maxig}

\subsubsection{MaxAM}
\label{subsubsection::maxam}

\subsubsection{MaxCHI}
\label{subsubsection::maxchi}

\subsubsection{MaxGSS}
\label{subsubsection::maxgss}

\subsubsection{MaxCC}
\label{subsubsection::maxcc}

\subsubsection{MaxOR}
\label{subsubsection::maxor}

\subsubsection{MaxCTD}
\label{subsubsection::maxctd}

\subsubsection{MaxTFIDF}
\label{subsubsection::maxtfidf}

\subsubsection{MaxTFICF} 
\label{subsubsection::maxtficf}

\cite{ChihHow04}




\section{Credibilidade baseada em Grafos}
\label{sec::pg_cred_baseada_grafos}

\subsection{Métricas modeladas.}
\label{subsec::pg_metricas_grafos}

asdf

